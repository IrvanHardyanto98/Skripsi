%versi 2 (8-10-2016) 
\chapter{Pendahuluan}
\label{chap:intro}
   
\section{Latar Belakang}
\label{sec:latar belakang}
Meningkatnya penggunaan layanan berbasis lokasi menyebabkan jumlah dataset, yang disimpan dalam bentuk trajektori, terus-menerus bertambah dalam waktu yang singkat. Trajektori adalah urutan rekaman lokasi pada waktu tertentu dari benda yang bergerak. Tidak hanya jumlah dataset yang terus bertambah, jumlah layanan berbasis online yang menyediakan streaming data spario-temporal juga terus bertambah. Contohnya adalah AccuTracking yang membantu pemilik toko dan perusahaan pengiriman barang untuk melacak posisi kendaraan pengangkut barang secara online.

\section{Rumusan Masalah}
\label{sec:rumusan}
Bagian ini akan diisi dengan penajaman dari masalah-masalah yang sudah diidentifikasi di bagian sebelumnya. 

\dtext{6}

\section{Tujuan}
\label{sec:tujuan}
Akan dipaparkan secara lebih terperinci dan tersturkur apa yang menjadi tujuan pembuatan template skripsi ini

\dtext{7}

\section{Batasan Masalah}
\label{sec:batasan}
Untuk mempermudah pembuatan template ini, tentu ada hal-hal yang harus dibatasi, misalnya saja bahwa template ini bukan berupa style \LaTeX{} pada umumnya (dengan alasannya karena belum mampu jika diminta membuat seperti itu)

\dtext{8}

\section{Metodologi}
\label{sec:metlit}
Tentunya akan diisi dengan metodologi yang serius sehingga templatenya terkesan lebih serius.

\dtext{9}

\section{Sistematika Pembahasan}
\label{sec:sispem}
Rencananya Bab 2 akan berisi petunjuk penggunaan template dan dasar-dasar \LaTeX.
Mungkin bab 3,4,5 dapt diisi oleh ketiga jurusan, misalnya peraturan dasar skripsi atau pedoman penulisan, tentu jika berkenan.
Bab 6 akan diisi dengan kesimpulan, bahwa membuat template ini ternyata sungguh menghabiskan banyak waktu.

\dtext{10}